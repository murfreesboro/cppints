%
% 
%
%%%%%%%%%%%%%%%%%%%%%%%%%%%%%%%%%%%%%%%%%%%%%%%%%%%%%%%%%%%%%%%%%%%%%%%%
\chapter{Fundamental Components in CPPINTS}
%
%
%
In this chapter the fundamental component classes in CPPINTS will be discussed in 
detail. These classes forms foundation for building the recurrence
relation(RR).

\section{Basis Set}
%
% 1  what's the form of basis sets?
% 2  why in basis set class we only store l,m,n information?
%
\label{bs}

In quantum chemistry, the basis set functions is the fundamental 
unit for practical calculations\cite{Davidson_Feller_CR_86_681_1986}. 
Primarily the basis set functions can be divided into radial part
and angular part. The radial part for the basis set function we discuss
in CPPINTS is formed by Gaussian functions, and the angular part could 
be further divided into two groups in terms of its function form.  
One group uses the spherical harmonics:
\begin{equation}
 \chi = Y_{m}^{L}(\theta,\phi)e^{-\alpha r^{2}}
\end{equation}
The other group uses the Cartesian function:
\begin{equation}\label{basis_set_cart_form}
 \chi = x^{l}y^{m}z^{n}e^{-\alpha r^{2}}
\end{equation}
In this program, we only focus on the Cartesian type
Gaussian basis set functions.

Generally the basis set function $\psi$ is a linear combination of 
Gaussian functions, and each such Gaussian function is also termed
as primitive function:
\begin{equation}\label{program_contract_basis_set}
	\psi = \sum_{\mu}d_{\mu}\chi_{\mu}
\end{equation}
$d$ is pre-optimized contraction coefficients.
$\chi_{\mu}$ is the primitive 
functions as defined in \ref{basis_set_cart_form}.
All of $\chi$ are on the same center as $\psi$, and they all share the 
same angular momentum with the basis set function.

For each basis set function, $x^{l}y^{m}z^{n}$ is its angular momentum part,
which is characterized by three number of l, m, and n. The $e^{-\alpha r^{2}}$
is its radial part, so l,m,n combined with exponential factor $\alpha$ and 
its coefficient of $d_{\mu}$; that give all of information to get $\psi$.

In the recurrence relation(RR), typically it starts with the bottom integral
\footnote{for VRR, bottom integral is in form of $(00|00)^{(m)}$ etc. For 
HRR, bottom integral is in form of $(0e|0f)$. See the chapter discussing RR 
for more information} and by raising up angular momentums it reaches the target
integrals. Therefore in the basis set class, we only use $l,m,n$ to represent 
the basis set.

In CPPINTS we also define the ``NULL'' basis set which means that 
this basis set is physical meaningless (l, m and n are all set to be -1). 
This type of basis set is used to complete an integral definition (see the 
\ref{integral} for more definition).

The basis set is defined in the file basis.cpp and basis.h.

%%%%%%%%%%%%%%%%%%%%%%%%%%%%%%%%%%%%%%%%%%%%%%%%%%%%%%%%%%%%%%%%%%%%%%%%
\section{Shell}
%
% 1  what's the shell? Why shell only have one data member of L?
% 2  what is the basis set order? How can we change it?
%
\label{shell}

In quantum chemistry, shell is an aggregation of same type of basis set
functions; whose total angular momentum is same(sum of l,m,n). For example, 
P shell contains 3 basis set functions, their l, m, n are characterized by:
\begin{align}\label{pshell_example}
	P_{x} &\Leftrightarrow (1,0,0) \nonumber \\
	P_{y} &\Leftrightarrow (0,1,0) \nonumber \\
	P_{z} &\Leftrightarrow (0,0,1)
\end{align}
By following the same principle, it's able the form D shell ($L = l+m+n = 2$), 
F shell ($L = l+m+n = 3$) etc.

As what we have demonstrated in the basis set functions section, to fulfill
recurrence relation it only requires the angular momentum information. 
Therefore, in the shell class it's only the total angular momentum L
is defined.

Here it needs to emphasize that how to arrange the basis set
functions in a given shell, this is called ``basis set order''.
For example, for the P shell, it's able to have:
\begin{equation}
 P_{x} \Rightarrow P_{y} \Rightarrow P_{z}
\end{equation} 
or the other order of 
\begin{equation}
 P_{z} \Rightarrow P_{y} \Rightarrow P_{x}
\end{equation} 

Since different basis sets in a shell are in a same level, therefore
there's not a basis set order prior to the others. It's able to 
pick up any basis set order theoretically. In this program, we 
pick up the ``libint'' order (which is used by libint program
\footnote{\url{http://sourceforge.net/projects/libint/}}). 
In basisutil.h, the arrangement of basis set order is given for 
L up to 20. On the other hand, it's able to use the other 
type of basis set orders. Therefore we separate the basis set 
order information all into the basisutil.h and basisutil.cpp.

In terms of a given basis set order, each basis set occupies an
unique position in the 
basis set order array, which refers as ``local index'' of the 
basis set in the given shell. For example, in the above P shell
case \ref{pshell_example} $Px$ is 0, $Py$ is 1, $Pz$ is 2. For 
each basis set it's able to set up some ``ONE-TO-ONE'' mapping 
relation between basis set and it's position in the basis set 
order list. Such fundamental relation is used for building the 
mapping relation between shell quartet and integral (see 
section \ref{mapping_integral_sq} for more details).

If the user wants to employ other type of basis set order other 
than the libint order, here is the procedure the user should
follow:
\begin{itemize}
 \item generate all of explicit basis set order array and 
	 replace the old content of ``LIBINT\_BASIS\_SET\_ORDER''
	 with the new array (this is in basisutil.h). This step
	 will enable you to get correct basis set by given 
	 a basis set index;
 \item Secondly the function of ``getLocalBasisSetIndex'' in
	 basisutil.cpp should be updated for the new basis 
	 set order. This step will give the correct local index
	 by an arbitrary input l, m, n value of basis set.
\end{itemize}
Now the mapping between index and angular momentum should 
set up. You should get correct integral results without 
referring to the other places.

Similar with basis set, we also define ``NULL'' shell whose 
L is set to be -1. This type of shell is used to complete 
shell quartet definition.

The shell is defined in shell.h and shell.cpp.

%%%%%%%%%%%%%%%%%%%%%%%%%%%%%%%%%%%%%%%%%%%%%%%%%%%%%%%%%%%%%%%%%%%%%%%%
\subsection{Composite Shells}
%
% 1  what is the composite shell?
% 2  how we handle the composite shell? 
\label{composite_shell}
%
%
The composite shell is that for each shell it may contain several
type of sub-shells. For example, SP shell has one S shell and one 
P shell, and SPD shell has S shell, P shell and a D shell. All of 
these sub-shells share the same exponential factors for it's radial part,
and each of them has its own contraction coefficients of $d$ in equation
\ref{program_contract_basis_set}. The most famous basis set library using 
composite shell is Pople basis sets (6-31G etc.)

In our program, the shell class is defined for ``pure'' shell(shell that
only corresponds to one $L$ value) and the composite shell situation is 
handled elsewhere. You can refer to the section \ref{composite_shell_quartet} 
etc. for more details that how CPPINTS handle the composite shell situation.

%%%%%%%%%%%%%%%%%%%%%%%%%%%%%%%%%%%%%%%%%%%%%%%%%%%%%%%%%%%%%%%%%%%%%%%%
\section{Integral Type}
%
% how to represent the integral operator?
%
\label{inttype}

In CPPINTS, we have a series of pre-defined integer to represent 
the ``OPERATORS'' used in quantum chemistry integrals. For example,
the two body overlap integral(TOV), nuclear-electron attraction integral(NAI)
etc. These integers are defined in the inttype.h and inttype.cpp(it's 
name appears as macro defined in general.h). The pre-defined integral 
is a data member for both integral and shell quartet class.

For the given integral type, there are some integral properties that 
is solely determined by the integral operator itself. For example,
NAI is two body integral, and it requires $(0|0)^{(m)}$ for RR etc.
You can find the details from inttype.cpp that what kind of properties 
can be determined only from operator itself.

%%%%%%%%%%%%%%%%%%%%%%%%%%%%%%%%%%%%%%%%%%%%%%%%%%%%%%%%%%%%%%%%%%%%%%%%
\section{Integral and Shell Quartets}
%
% 
%
%%%%%%%%%%%%%%%%%%%%%%%%%%%%%%%%%%%%%%%%%%%%%%%%%%%
\subsection{Integral}
\label{integral}
%
% 1  what is integral?
% 2  how to represent it?
%
For a given operator representing quantum quantity (for example,
the kinetic operator, electron repulsion operator etc.) it's able 
to form the integrals based on the basis set functions:
\begin{equation}
 I = \langle \psi_{l_{1}m_{1}n_{1}}(r)\psi_{l_{2}m_{2}n_{2}}(r)| 
 \hat{f}(r,r^{'})| \psi_{l_{3}m_{3}n_{3}}(r^{'})
 \psi_{l_{4}m_{4}n_{4}}(r^{'})\rangle
\end{equation}
therefore to define an integral, basically the following 
information is needed:
\begin{itemize}
 \item operator;
 \item basis set information for all given integral positions
\end{itemize}
Because recurrence relation also uses the integrals $(ab|cd)^{(m)}$,
an additional $m$ value is defined in both integral and shell
quartet class.

Primarily the integrals could be divided into different categories according 
to the number of its basis set function components. In quantum chemistry, 
the possible number of basis set functions in the integral is ranging 
from 1 to 4. Therefore, integral is ranging from one body integral to 
four body integrals. 

The four possible basis set positions are called as: ``BRA1'', ``BRA2'',
``KET1'' and ``KET2''. For one body integral, it's only the bra1
position having basis sets(other positions have null basis set); 
for the two body integral, it's only bra1 and bra2 positions having 
basis sets; for the three body integral, 
it's only bra1, bra2, and ket1 positions having basis sets. 
Such position definition is used across the whole program, and the 
similar definition holds for shell quartet, too.

The integral class always has four position for the basis set (see 
integral.h). For the one to three body integrals, we use NULL
basis set to complete the integral definition.

The integral is defined in integral.cpp and integral.h.
%%%%%%%%%%%%%%%%%%%%%%%%%%%%%%%%%%%%%%%%%%%%%%%%%%%
\subsection{Shell Quartet}
%
% how to define shell quartet?
% what is the relation between shell quartet and integral?
%
\label{shell_quartet}
Similar to the relation between basis set and shell, the aggregation 
of certain type of integrals will lead to the ``shell quartet''.

For example, for the electron repulsion operator it's able to define 
the shell quartet over four shells:
\begin{equation}
 SQ_{P,P,P,P} = \langle P_{bra1}P_{bra2}|
 \frac{1}{r_{12}}|P_{ket1}P_{ket2} \rangle
\end{equation}
This shell quartet includes all of integrals in terms of basis sets
for the given four P shells. Because each P shell have 3 basis set
functions, the shell quartet contains $3^{4} = 81$ ERI integrals.

Similar to the integral class which is defined based on the basis set
class; the shell quartet is constructed based on the shell class. Therefore
for defining a fundamental shell quartet, it needs shell information
on all of four BRA1, BRA2, KET1 and KET2 positions (null shell is used
to complement the shell quartet definition) as well as the operation
information.

The definition of shell quartet could be referred to shellquartet.h
and shellquartet.cpp.

%%%%%%%%%%%%%%%%%%%%%%%%%%%%%%%%%%%%%%%%%%%%%%%%%%%
\subsection{Mapping between Integral and Shell Quartet}
\label{mapping_integral_sq}

For CPPINTS, shell quartet is the basic unit for deriving the 
recurrence relation. The integrals are considered to be attached to 
its corresponding shell quartet.

Based on the ``ONE-TO-ONE'' mapping relation between basis set 
and shell, it's able to construct the ``ONE-TO-ONE'' mapping relation
between integral and its shell quartet. For each integral, it 
has a unique position according to the given basis set order.
For example, integrals in the shell quartet $(SPSP|SPSP)$ may 
correspond to an order like:
\begin{align}
 (SS|SS)         &\rightarrow  0 \nonumber \\
 (P_{x}S|SS)     &\rightarrow  1 \nonumber \\
 (P_{y}S|SS)     &\rightarrow  2 \nonumber \\
 (P_{z}S|SS)     &\rightarrow  3 \nonumber \\
 (SP_{x}|SS)     &\rightarrow  4 \nonumber \\
 (P_{x}P_{x}|SS) &\rightarrow  5 \nonumber \\
 (P_{y}P_{x}|SS) &\rightarrow  6 \nonumber \\
 (P_{z}P_{x}|SS) &\rightarrow  7 \nonumber \\
                 &\cdots
\end{align}
By giving the shell quartet and this unique position, it's able 
to reconstruct the integral; on the other hand, from the given 
integral it's able to derive its unique position\footnote{please
refer to integral.cpp for more details that how we do it}.

%%%%%%%%%%%%%%%%%%%%%%%%%%%%%%%%%%%%%%%%%%%%%%%%%%%
\subsection{Representation of Integral in Recurrence Relation}
%
% why we use position ID to reprernt integral in shell quartet?
%
\label{representation_integral_sq}

The ``ONE-TO-ONE'' mapping relation could be used to improve the memory 
usage for CPPINTS. Suggest we have a $(II|II)$ ERI shell 
quartet for deriving its vertical recurrence relation, to record
all of the information for each integrals requires the following 
items:
\begin{itemize}
 \item LHS integral;
 \item RHS integral and its coefficients.
\end{itemize}

For each integral, to characterize it it requires $3\times 4 = 12$
integers to record angular momentum information, plus two integers
for operator and m value. Therefore each integral needs 14 integer
\footnote{we have not considered the division yet, which is related to
composite shell quartet case}.

For the OS VRR framework, the recursive relation has 8 items on the 
RHS so there are totally 9 integrals, which needs 126 integers. 
Because the $(II|II)$ has $28^4$ integrals, it needs 77446656 integers.
Suggest each integer takes 4 byte memory, the total memory for 
storing integer of this shell quartet would be 296mb! So far we did not
count in the coefficients.

However, if we use the ``position ID'' to represent the integrals 
inside the shell quartet, then each integral only need one integer 
thus it's only 5.3mb memory needed. Therefore, to use the position 
ID to represent integrals inside shell quartet is a good way for 
memory saving with only marginal CPU cost added. This is the way
we handle integrals in the recurrence relation(RR) process.

%%%%%%%%%%%%%%%%%%%%%%%%%%%%%%%%%%%%%%%%%%%%%%%%%%%
\subsection{Composite Shell Quartet}
%
%
\label{composite_shell_quartet}

In a given shell quartet, if one of its shell is in composite type; 
then this shell quartet is composite shell quartet. It contains
several shell quartets in terms of pure shell. For example,
shell quartet $(DD|DSP)$ has two pure shell quartets; namely 
$(DD|DS)$ and $(DD|DP)$.

How to process composite shell quartet in the recurrence relation?
It's depending on whether it's HRR or it's in VRR.

For VRR, Let's take ERI as example. Because the recurrence relation 
is generally expressed as: 
\begin{align}
I(L,m) &= a_{0}I_{0}(L-1,m) + a_{1}I_{1}(L-1,m+1) \nonumber \\ 
&+ a_{2}I_{2}(L-2,m) - a_{3}I_{3}(L-2,m+1) \nonumber \\
&+ a_{4}I_{4}(L-2,m) - a_{5}I_{5}(L-2,m+1) \nonumber \\
&+ a_{6}I_{6}(L-2,m+1) + a_{7}I_{7}(L-2,m+1)
\end{align}
The above integral expression is on primitive Gaussian function, and 
$a_{i}$ is RR coefficient which is independent with the integral.
Because VRR is linear with the contraction coefficients $d$,
the expression above holds for both contracted and un-contracted primitive
Gaussian functions. Therefore, the recurrence relation for VRR is 
independent of the contraction information. We can simply get the 
integral with contraction coefficients by multiply it to the final
results:
\begin{equation}\label{composte_sq_contraction_coefs}
 I_{contracted}(L,m) = d_{bra1}d_{bra2}d_{ket1}d_{ket2}I(L,m)
\end{equation}
This is what we do for dealing with the composite shell quartets in VRR.
Because in composite shell the sub-shells share the same exponential
factor, the raw integral $I(L,m)$ in \ref{composte_sq_contraction_coefs} 
is same between different pure shell quartets; the final integral results
with contraction coefficients could be simply retrieved by performing 
\ref{composte_sq_contraction_coefs}. This is called ``contraction'' step
in VRR. The pseudocode is given like this:
\begin{verbatim}
loop over ket side primitive Gaussian pairs:
  loop over bra side primitive Gaussian pairs:
    compute bottom integrals;
    do VRR to derive result integrals;
    perform contraction step; 
  end loop
end loop
\end{verbatim}
For pure shell quartet like $(DD|DD)$, there's no need to perform 
contraction step because it can simply add contraction information
to the bottom integrals $(00|00)^{(m)}$ so that VRR is performed 
for contracted integrals. After VRR is ending, the result integrals
get updated by the result in the primitive integral loops.

However, for HRR the story is different. Because HRR is usually applied on
contracted integral results (it sums over all of integrals on 
primitive Gaussian functions), the equation \ref{composite_sq_hrr}
shows that how we can step from
un-contracted integrals to contracted ones in HRR:
\begin{align}\label{composite_sq_hrr}
 [a(b+\iota_{i})|cd]_{k} &= [(a+\iota_{i})b|cd]_{k} + 
(A_{i} - B_{i})[ab|cd]_{k} \rightarrow \nonumber \\
\sum_{k}C_{k}[a(b+\iota_{i})|cd]_{k} &= \sum_{k}C_{k}[(a+\iota_{i})b|cd]_{k} + 
(A_{i} - B_{i})\sum_{k}C_{k}[ab|cd]_{k} \rightarrow \nonumber \\
(a(b+\iota_{i})|cd) &= ((a+\iota_{i})b|cd) + 
(A_{i} - B_{i})(ab|cd)
\end{align}
Here $[ab|cd]$ is used to represent the integrals on un-contracted Gaussian 
functions, and $(ab|cd)$ for the contracted integrals; $C_{k}$ is the 
combination for all of contraction coefficients of $d$.

It's clear from \ref{composite_sq_hrr} that as long as LHS and RHS share the 
same contraction coefficients, HRR can be applied from $(0e|0f)$ until
the final target integral $(ab|cd)$ is derived. Hence for the composite 
shell quartets, each pure shell quartet has its own HRR process and they
are independent with each other.

For this reason, we have a data member named as ``division'' on both
integral and shell quartet classes. It's an ID for each pure shell quartet
in the composite shell quartet. For example; $(DSP|DSP)$ has four shell quartets;
$(DS|DS)$, $(DS|DP)$, $(DP|DS)$ and $(DP|DP)$ and each pure shell quartet has it's 
own ID number (characterized by division). They will have their own
HRR process.

%%%%%%%%%%%%%%%%%%%%%%%%%%%%%%%%%%%%%%%%%%%%%%%%%%%
\subsection{Sorting Shell Quartets}
%
%
\label{sort_shell_quartet}

After RR(VRR or HRR) path is constructed in detail, a sorting 
function is needed to be applied to the whole LHS shell quartets
so that to make sure all of RHS shell quartets are well defined
in the previous content. Therefore, we need sorting function for 
building the correct RR.

The inconvertibility character for VRR establishes in section 
\ref{optimal_path} in fact enables us to set up sorting facility
on the shell quartet in the RR direction. In other words, in 
accumulating all of RR terms along the VRR path; we are able 
to sort the shell quartets either from bottom integrals to
the results(this is always from RHS to LHS in terms of RR),
or from the results to the bottom integrals in reverse.

This is what the operator $<$ function in shell quartet class
does. By following the VRR formula\footnote{right now the function
just follows the VRR in OS framework}, it's able to distinguish
the shell quartet on either RHS or LHS. More details can be 
found in the comments of this function.

HRR does not hold the inconvertibility character as shown in 
section \ref{optimal_path}. However, it's still applicable 
to set up some algorithm to sort the shell quartets in the 
HRR process. The sorting function for HRR is presented in
hrrCompare function in shellquartet.cpp. Therefore, it's 
able to sort the HRR shell quartets so that we print out 
the HRR formula in correct order.

