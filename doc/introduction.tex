%
% 
%
\chapter{Introduction}

CPPINTS is a general program to generate analytical integrals based on Gaussian type primitive 
functions for quantum chemistry program.

In version 1 the codes is able to generate the following types integrals in terms of energy 
calculation:
\begin{itemize}
 \item two and three body overlap integral(OVI);
 \item two and three body kinetic integral(KI);
 \item two, three and four body electron repulsion integral(ERI);
 \item two body nuclear attraction integral(NAI);
 \item momentum integral(MOM);
 \item integrals used in electrostatic potential(ESP)
\end{itemize}

The integral code can support both single and double floating precision.

\section{General Notations for Integral on Gaussian Functions}

\section{Code Structure}

\section{Change Log}

\section{Notes and Limitations for Version 1}
\label{v1_notes_limits}

Additional notes for version 1:
\begin{itemize}
 \item the up recursive process for $SSSS^{m}$ integrals could be
  only applied to double variable. The float accuracy
  is not good. Therefore even if the with\_single\_precision
  is defined, we still use double type to compute $SSSS^{m}$
  integrals.
 \item the implementation for the derivatives information on
  doing the kinetic energy (three body kinetic) so far
  is not quite clean. It's more like the "unfinished"
  idea and later we need to go back to revise it here(will be revised
  in version 2).
 \item for the three body kinetic integrals, since it's only
  can be done within VRR, therefore there's angular
  momentum limit imposed in terms of the calculation
  capacity. Currently we can see that for ``(h,h,h)''
  shell quartet it will fail because too many shell
  quartets in searching for optimum shell quartet path.
\end{itemize}

potential problems for the version 1:

\begin{itemize}
 \item Because of the algorithm used in HRR comparison,
  the SPD shell case can not be used in this program.
  It could give wrong results(see the comment in the
  hrrCompare function in shellquartet.cpp).
  therefore only SP shell is included so far.
 \item for the sqintsprint.cpp, we have a function nTotalInts()
  to compute all of integral number for input rrsqlist.
  this is needed for ESP. However, in general at the
  VRR contraction, we do not know whether we can still
  use nTotalInts() as additional offset? This may be
  wrong for the following case:
  abcd is in dimension of (nbas, nbas, additional\_dimension, nbas)
  in this case, additional dimension is in the middle
  of results. In this case, the contraction will be
  wrong for computing offset.
  We need to keep an eye on this possible situation.
 \item for the rrints.cpp file, the print function; as we
  print the RHS of shell quartet; we can not determine
  the position information. The position information
  is useful for the derivatives calculation. In the future,
  we may need to add in the derivatives information into
  the shell quartet.
  the current setting may get a wrong result for the
  situation that the lhs shell quartet and rhs shell
  quartet are both with derrivatives symbol.
\end{itemize}

\section{License Agreement}

CPPINTS use MIT license as below:

\newpage

\begin{verbatim}
 
The MIT License (MIT)

Copyright (C) 2015 The State University of New York at Buffalo

Permission is hereby granted, free of charge, to any 
person obtaining a copy of this software and associated 
documentation files (the "Software"), to deal in the 
Software without restriction, including without limitation 
the rights to use, copy, modify, merge, publish, distribute, 
sublicense, and/or sell copies of the Software, and to permit 
persons to whom the Software is furnished to do so, subject 
to the following conditions:

The above copyright notice and this permission notice shall 
be included in all copies or substantial portions of the Software.

THE SOFTWARE IS PROVIDED "AS IS", WITHOUT WARRANTY OF ANY KIND, 
EXPRESS OR IMPLIED, INCLUDING BUT NOT LIMITED TO THE WARRANTIES 
OF MERCHANTABILITY, FITNESS FOR A PARTICULAR PURPOSE AND 
NONINFRINGEMENT. IN NO EVENT SHALL THE AUTHORS OR COPYRIGHT 
HOLDERS BE LIABLE FOR ANY CLAIM, DAMAGES OR OTHER LIABILITY, 
WHETHER IN AN ACTION OF CONTRACT, TORT OR OTHERWISE, ARISING FROM,
OUT OF OR IN CONNECTION WITH THE SOFTWARE OR THE USE OR OTHER 
DEALINGS IN THE SOFTWARE.
\end{verbatim}