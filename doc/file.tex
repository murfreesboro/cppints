


\chapter{Code Printing in CPPINTS}

For an arbitrary analytical shell quartet:
\begin{equation}
 I_{ijkl} = \int \phi_{i}(r)\phi_{j}(r)\hat{f}(r,r^{'})\phi_{k}(r^{'})\phi_{l}(r^{'}) dr dr^{'}
\end{equation}
There are five possible code sections:
\begin{itemize}
 \item VRR;
 \item HRR1;
 \item HRR2;
 \item non-RR;
 \item derivative
\end{itemize}
These code sections has been discussed in detail in previous chapters. After all of 
the code sections are generated, now the question is: how to link them together 
and print them out into one or a group of compilable files? This is the topic
of this chapter.

\section{Guidelines for Code Printing}
\label{code_generation_guide}

there are a couple of instructions for the code generation:
\begin{itemize}
 \item each file only has one function defined;
 \item the code will be organized in the following ways:
 \begin{itemize}
  \item only one code file generated. In this case all of code sections are 
  printed out into this code file.
  \item a group of code files generated. In this case there has a ``top file''
  where the driver function is defined inside, all of other functions will be 
  called by this driver function. The other files are ``sub files'' where
  the working functions are defined. Each code module may correspond to one 
  or more sub files, or printed out in the top file;
 \end{itemize}
 \item for each file, if the integrals are locally defined and used, then the 
 corresponding shell quartet will be printed out in variable form rather 
 than the array form;
 \item In the multiple files case, the input and output shell quartets for 
 each function are all in array form;
 \item all of array declaration should be in top file
\end{itemize}


\section{Algorithm for Code Printing Setup}

\subsection{Analyze User's Requirement}
\label{code_generation_user}

There are several printing related parameters defined in the infor class\ref{infor_sqintsinfor}, 
so that user is able to control the code printing:
\begin{itemize}
 \item nLHSForVRRSplit;
 \item nLHSForHRR1Split;
  \item nLHSForHRR2Split;
 \item nLHSForNonRRSplit;
 \item nLHSForDerivSplit;
 \item nLHSForTopFileSplit
\end{itemize}
These parameters are designed to control the sub files generation 
for the corresponding VRR, HRR, non-RR and derivatives code sections, and the last
parameter is used to control the size of top file.

Here we use the number of LHS integrals as estimation for determining file split.
The reason that every code section has its own parameter defined, is because
different code sections have different RHS length. For example, for VRR the 
LHS ERI may have 5 integrals on the RHS, however for HRR the RHS is always
composed by 2 integrals. Therefore each module has it's own limit to determine
whether file split is executed.

In general, if the total LHS integrals for the given code section is larger 
than the limit value, the result code section will be broken into one or more sub files.
Each sub file should have LHS integrals less than the given code module limit.

Here for the case that the LHS integral number is less than 1.5 times of the given
code module limit, we will not split the corresponding code sections into two files
so that to keep the integrity of the code section. The scale limit of 1.5 is a 
constant in the infor class. Therefore please be aware, that do not set the 
LHS integrals limit too high so that to make the compilation become difficult.

Finally, the parameter of nLHSForTopFileSplit is to restrict the size of top file.
There's possible that every code section is smaller than the corresponding limit,
but the result top file is too large to compile. Therefore we will count the 
total number LHS integrals from all code sections to see whether it's larger than
the nLHSForTopFileSplit, if so the file split will be performed for the code sections
until the number of LHS integrals in top file is smaller than the limit.

For example, the derivatives calculation for an ERI generates 60000 LHS integrals, while
the nLHSForTopFileSplit is set to 50000. According to the above principle we will take
the last derivative section out into sub file, if the size limit is not met then accordingly
to take sections HRR2, HRR1 etc. out until the size limit is met.

\subsection{Code Printing Setup for Every Code Section}
\label{code_generation_print}

The core idea for set up printing codes is concentrating on two questions:
\begin{itemize}
 \item how to determine the file split for every code section;
 \item for every LHS and RHS shell quartet, we use array or variable
 form to express the integral?
\end{itemize}
both of the two features are closely related. file split determination
is the prior step, and the final printing procedure requires each LHS and RHS 
shell quartets know it's array/variable status.

According to the above instruction, in CPPINTS we are trying to make the integrals 
as much as in variable form. The variable form of integral is locally generated on
the stack frames, therefore it's faster operated comparing with the heap memories.

However, it's local variables can be made with variable form. For the integrals
passing in/out of function, they must be in array form. Therefore, it's after 
the file split status are determined for all of code sections, the variable/array
form determination can be processed for the LHS and RHS shell quartets.

Every code section is composed by a list of RRSQ \footnote{please
see the section of \ref{rrints}}. The process to form the sub files, 
is to break the RRSQ list for the given code section (VRR, HRR or non-RR etc.)
into several chunks, and each chunk corresponds to one sub file.

The procedure to set up the code printing is as follows:

\begin{step}
\textbf{Finish code generation for all of sections}
 
The code printing set up can be only made when all of code generation is finished. 
This is because the input shell quartets for the given code section is coming 
from previous sections (except VRR, where the VRR bottom integrals are always directly
computed). However, the code generation process must be in reverse order (from last section
to VRR), therefore the file status determination can be only made when all of 
code sections are finished.
\end{step}

\begin{step}
\textbf{determine chunks for RRSQ list in code sections in normal order}
 
The first step for printing set up is to divide the original RRSQ list into chunks according
to the user's specification. As a result each RRSQ LHS/RHS shell quartets will be either 
assigned a sub file, or clearly state they are in top file. This process is carried out for 
code sections in normal order, because in normal order the previous code section generates 
the output for the following code sections, so the input/output shell quartets status for 
each section is clearly defined. 
\end{step}

For the LHS/RHS shell quartets, they are in one of the following status(the status are
defined in general.h):
\begin{description}
 \item [SUB\_LOCAL\_SQ] the given shell quartet is ``local'' to the given Sub
 file, it's generated inside the sub file and used only inside the sub file, too.
 Therefore it's local, and the shell quartet will be printed out in variable 
 form according to the above instruction.
 \item [SECTION\_LOCAL\_INPUT\_SQ] the shell quartet is passing into the sub file, and 
 it's generated from previous sub file in the same section. For the 
 given code section, such shell quartet is ``local'' to the code section and not be 
 used in other code sections. However because the code section is divided into multiple 
 sub files therefore it's not local to the sub file.
 \item [SECTION\_LOCAL\_OUTPUT\_SQ] similarly with the \\
 ``SECTION\_LOCAL\_INPUT\_SQ'', this 
 shell quartet is section locally shell quartet, generated in the given sub file 
 but is going to pass over to following sub files for use.
 \item [SECTION\_INPUT\_SQ\_VAR\_FORM] the given shell quartet is passing from previous 
 sections, however because that section is not in file split so this shell quartet is 
 in variable form.
 \item [SECTION\_INPUT\_SQ\_ARRAY\_FORM] Similarly the shell quartet is passing from
 previous sections, but in array form.
 \item [SECTION\_OUTPUT\_SQ\_VAR\_FORM] The shell quartet is going to pass to following
 sections and with variable form.
 \item [SECTION\_OUTPUT\_SQ\_VAR\_FORM] The shell quartet is going to pass to following
 sections and with array form.
 \end{description}
 
 Because non-RR code sections does not have temporary variables, they does not have 
 the shell quartets locally to the section; so the non-RR shell quartets are all section
 input/output. As for it chooses array or variable form, it depends on 


After this step 

After all of code sections are generated completely, it's time to set up the 
file split status for every code section. This step will answer the following 
questions:
\begin{enumerate}
  \item for the given code section, whether it's print out in top file or printed 
 onto the sub files determined by the user set limit?
 \item how many sub files for the given code section? 
 \item for each sub file, what are the RRSQ sections involved in the sub file?
\end{enumerate}



The INPUT\_SQ is easy to determine. For the given chunk of RRSQ list, if the RHS is not defined
locally then it must be passed from elsewhere. The only exception is the bottom VRR integrals,
which are all directly computed.

However, to decide whether a given shell quartet is a ``local'' one or needed to be passed out,
is not something easy. Here there are two possible cases:
\begin{enumerate}
 \item firstly, this shell quartet may be used in other sub files but still within the same 
 code section;
 \item secondly, the shell quartets may be used in following code sections. They can serve
 as input for the next code section, or even serve as input for the the nonadjacent code 
 sections.
\end{enumerate}
for the given LHS shell quartet, after it's eliminated the possibility as being ``OUTPUT\_SQ'',
then it's status is defined as ``LOCAL\_SQ''. Such checking procedure will be carried out for every 
LHS shell quartets in each code section. Here it's worthy to note that the checking is 
performed in reverse order, which means it starts from the last code section until VRR.
After the LHS shell quartets is clearly marked, all of RHS shell quartets are updated
accordingly. 

After the shell quartet status is identified, the question that to use array or variable
form to print out the RR/non-RR expression is clearly solved. All of above information
will be finally updated into each RRSQ before printing.

\section{Class Design and Arrangement}

the classes worked for code printing is organized as follows:
\begin{description}
 \item [SQInts]  this is the top driver class to manipulate the code generation and 
 file printing;
 \item [infor]   this class handle the user specified parameters through input file;
 \item [SQIntsInfor]  This is the message class to pass original shell quartet, derivatives 
 and job information etc. to following work module;
 \item [SQIntsPrint]  This class gathers all of information related to printing;
 \item [VrrInfor] Because VRR has complicated bottom integrals computation, therefore VrrInfor
 class is set up to print the top loop structure and variables, bottom integrals, and contraction
 etc.
\end{description}

\subsection{SQInts}
\label{sqints}

SQInts is the commander of the whole code generation and printing process. It has two class 
members, one is SQIntsInfor, which directs the code generation process; the other is 
SQIntsPrint, and it works with RR and non-RR classes to monitor the code printing process.

Generally the SQInts is designed as the top level class, it's does not know any detail about 
the code generation; instead it uses SQIntsInfor to instruct the whole code generation. Next
SQInts oversees the code printing through SQIntsPrint; and this step will generate all of 
file pieces in the scratched directory.

Based on the existing file pieces, finally SQInts will merge the files together into the 
sub files and top file. This process is so general that SQInts only needs to know that 
what kind of file pieces exist. To merge file pieces into compilable files is the main
task of SQInts.

\subsection{Infor and SQIntsInfor}
\label{infor_sqintsinfor}

Infor class is designed to satisfy a simple purpose: recognize the user's requirement from the 
input file and initialize the user's specification for code generation and printing. The sample
of the user's input file can be found in the main directory.

As a component for SQInts, SQIntsInfor is a derived class of Infor; and it contains all of information
related to the input shell quartet and the job information. The information in SQIntsInfor concentrates
on the following aspects:
The information contained in the SQIntsInfor is formed in terms of the input
shell quartets, it mainly includes:
\begin{itemize}
 \item job order, which tells whether it's energy, first derivative or second derivatives job;
 \item operator definition;
 \item input shell quartet information;
 \item whether it's composite shell quartet? If so SQIntsInfor gives the offset, coefficients offset 
 etc. information for each pure shell quartet;
 \item derivatives information;
 \item code section information
\end{itemize}

\subsection{SQIntsPrint}
\label{sqintsprint}

The SQIntsPrint class another component of SQInts, and it is composed by two level structures:
\begin{center}
 SubFileRecord class $\Rightarrow$ CodeSectionRecord class $\Rightarrow$ SQIntsPrint
\end{center}

Primarily SQIntsPrint involves 5 classes member of CodeSectionRecord, each one corresponds to a 
possible code section. Inside the class of CodeSectionRecord, it contains a list of SubFileRecord
members and each one keep a sub file record. Totally the whole SQIntsPrint provides a complete
description for the file split status for the file generation process.

The set up of SQIntsPrint is for answering the two questions in section \ref{code_generation_print}:
how to set up file split for the code generation, and how to determine every shell quartet status 
from LHS and RHS. After all of RR and non-RR code generation, depending on the number of LHS integrals
the file split status, as well as input/output are set up for all of sub files and top file. 

Based on the file set up, LHS shell quartet status will be identified; and then status is updated
to all of RHS shell quartets for all code sections. Finally, the shell quartet status for both
LHS and RHS will be updated to the RRSQ data; then as printing the code, RRSQ will know how to express
the integral in array or in variable form.

Additionally it's worthy to note that for SQIntsPrint, it contains shell quartet array where these 
shell quartets are passed to nonadjacent code sections, and these nonadjacent code sections are in
sub file form. Therefore it requires that the shell quartets inside must be in array form. We only
have three types of array, namely for VRR, HRR1 and HRR2. This is because non-RR and derivatives 
section does not have nonadjacent sections. 

\subsection{VrrInfor}
\label{vrrinfor}

VrrInfor class is used to wrap up the details for forming the head of VRR, as well as VRR contraction
work. for forming the head of VRR, it concentrates on three aspects:
\begin{itemize}
 \item VRR loop structure initialization. This is looping over the primitive functions etc.
 \item VRR recurrence relation variables generation;
 \item VRR bottom integrals computation
\end{itemize}
VRRInfor also handles the function parameters generation for the case of VRR file split, and VRR 
contraction work. 

VRRInfor is an important class for modification if user wishes to add more type of integrals support.
Typically different type of integrals have different VRR structure, therefore the user may need 
to add in new functions to renew the VRR loop structure information, VRR variable information 
as well as corresponding bottom integral calculation. Also it's needed to keep an eye on the function
parameter generation for file split case, usually different type of integrals have different RR
variables and bottom integrals to pass; therefore the code over there is case dependent.


