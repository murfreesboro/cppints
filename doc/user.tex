%
% 
%
\chapter{User Manual}



\section{How to Compile CPPINTS}
%
% 1 c++ compiler
% 2 boost library as 3rd party library
% 3 Makefile
%
CPPINTS is a C++ program(that's the reason why it's named as ``CPPINTS''), 
therefore you need a C++ compiler. The intel C++ compiler icpc, and the 
GNU compiler g++ are both well tested to compile the codes. The result
integral codes are also C++/C source codes.

In addition, CPPINTS also relies on the boost library as third party 
library. You can download it at \url{http://www.boost.org/}. We use 
the boost File System\footnote{please refer to 
\url{http://www.boost.org/doc/libs/1_58_0/libs/filesystem/doc/index.htm}},
boost Math, boost Lexical Cast etc. modules. Except boost File system,
all of other modules are used as include templates therefore no static or 
dynamic libraries are needed.

The source code of CPPINTS is placed in the folder of ``src'', and the 
corresponding include files are placed in ``src/include''. To compile
the code is easy, there's a Makefile in the main folder so You 
can just type ``make'' in the main folder and the result binary 
will be generated. You can also modify the Makefile by yourself by 
selecting your C++ compiler and options. Please make sure that the library
of boost File System can be found and correctly linked. 

\section{How to Generate Integral Codes}
%
%
%
\label{generate_int_codes}
To generate the integral codes you simply run the following command:
\begin{center}
 ./cppints parameter\_file
\end{center}
After it's finished, a new folder containing the integral codes will be 
generated. For example, if you choose to use HGP scheme as HRR and OS
scheme as VRR, the result folder will be named as ``hgp\_os''.

There's a sample parameter file infor.txt placed in the main folder.
You can get details inside about the parameters which are used to direct the 
generation of integral code. For how to parse the parameter file,
you can refer to the section  \ref{infor_class} for more information.

\section{Use of Utility Codes}
\label{use_util_codes}
%
%
%
The ``util'' folder provides additional functions to complete or check
the generated integral codes. Each function is in a separate folder or 
file.
\begin{description}
 \item [constants] This folder is to generate the constants.h file included
 by the result integral codes. It defines mathematical constants used in generating
 the bottom integrals $(00|00)^{(m)}$ (please refer to the section \ref{fmt}
 for more details). If you want to compile the result integral codes, you 
 need to place the constants.h in your head file folder.
 \item [sssstest.cpp] This file is to compare the accuracy of different
 ways to calculate bottom integrals $(00|00)^{(m)}$, now it's stored only
 for archive purpose. 
 \item [fmt\_test] This folder stores the trial test files and it's corresponding log
 files for the hybrid scheme to calculate $f_{m}(t)$ function in section \ref{fmt}.
 Please see the section for more details.
 \item [norm\_compare] This folder contains files for experimental test which explores
 how normalization factor varies in accordance with change in angular momentum $L$. 
 See the README for more details.
 \item [headfiles] This folder contains python files which are used to generate 
 the entry function of the integral module as well as head files to record the whole 
 integral function prototype. See the following context and the python codes 
 for more information.
 \item [reorganizedcpp.pl] after the entry function is generated, this perl file
 may re-organize it by sorting the integrals functions in terms of $L_{max}$. $L_{max}$
 is the maximum $L$ among all shell components for the shell quartet.
 \item [efficiency] This folder holds python files to evaluate the FLOPS as well as
 shell quartet path for VRR and HRR in terms of each integral function. We note that
 to use the python code here the integral code can not be split, that is to say;
 the split mode can not be used for the integral codes if evaluation is applied.
 \item [makefiles] The python files here is used to generate Makefile or CMakeLists.txt 
 file(right now we do not have this function yet).
 \item [mem\_stat.py] This python file is used to generate the memory usage statistic data.
 This is useful when the user has memory manipulation class passing into the integral
 code and allocate the memory through the tool class. For example, we have a class named as 
 LocalMemScr, it's used to allocate memory for the array used in integral code. The code
 here can help you to know how much memory you need for dealing with the maximum memory
 requirement.
\end{description}
Here we need to note, that the codes here may need to be further revised before going into
use. Therefore, use it with caution!

The entry function generated in headfiles folder is a function formed by a large
switch statement, by comparing the input $L$ code it will try switching to the 
corresponding integral function. Here is an example from entry function of 
two body kinetic integrals:
\begin{verbatim}
void hgp_os_kinetic(const LInt& LCode, const UInt& inp2, 
const Double* icoe, const Double* iexp, const Double* iexpdiff, 
const Double* ifac, const Double* P, const Double* A, 
const Double* B, Double* abcd)
{
  switch (LCode) {
    case 100001:
      hgp_os_kinetic_p_sp(inp2,icoe,iexp,iexpdiff,
      ifac,P,A,B,abcd);
      break;
    case 3003:
      hgp_os_kinetic_f_f(inp2,icoe,iexp,iexpdiff,
      ifac,P,A,B,abcd);
      break;
    case 4:
      hgp_os_kinetic_g_s(inp2,icoe,iexp,iexpdiff,
      ifac,P,A,B,abcd);
      break;
    case 3005:
      hgp_os_kinetic_h_f(inp2,icoe,iexp,iexpdiff,
      ifac,P,A,B,abcd);
      break;
    case 100003:
      hgp_os_kinetic_f_sp(inp2,icoe,iexp,iexpdiff,
      ifac,P,A,B,abcd);
      break;
    case 5005:
      hgp_os_kinetic_h_h(inp2,icoe,iexp,iexpdiff,
      ifac,P,A,B,abcd);
      break;
    case 2004:
      hgp_os_kinetic_g_d(inp2,icoe,iexp,iexpdiff,
      ifac,P,A,B,abcd);
      break;
    case 1001:
      hgp_os_kinetic_p_p(inp2,icoe,iexp,iexpdiff,
      ifac,P,A,B,abcd);
      break;
    .......
}
\end{verbatim} 
with entry function and head file recording the prototype of each integral
function, it's able to integrate the integral code into your project.

\section{How to Test Integral Codes}
%
%
%
The CPPINTS project has a folder named as ``test'', this is the folder
used to test the generated integral codes. It compares the generated
integral codes with direct calculation in terms of primitive Gaussian
functions. Right now the following integrals can be tested in this 
folder:
\begin{itemize}
 \item two and three body overlap integral;
 \item two and three body kinetic integral;
 \item nuclear attraction integral;
 \item electron repulsion integral (the direct calculation code is taken from
 libint package);
 \item moment integrals
\end{itemize}
 
For each type integral calculation it has a file to perform the comparison
task(the only exception is ERI. eri.cpp is the direct calculation code from
libint package, and eritest.cpp is the main file to do the task). By digging into
the work module you can get more details about the comparison.

Because the testing code port in the ERI code from libint package, to use the 
code it requires the MPFR\footnote{please see \url{http://www.mpfr.org/}} and 
GMP\footnote{please see \url{https://gmplib.org/}} library. Please make sure
they are installed so that to correctly compile the code. More details could
be referred from the Makefile in the test folder.

\section{How to Port Generated Integral Codes into Other Package}
%
%
%
Different from other integral package like libint, CPPINTS does not provide
the integral code in form of library; instead it directly generates the 
integral code and the user need to plug the raw integral code into their
coding system to use it.

The first step is to generate the integral code. User can refer to the 
section \ref{generate_int_codes}. After the code is generated, next step
is to generate entry function and head files by using the ``headfiles''
module in the util folder(please refer to \ref{use_util_codes} for more
information).

To use the integral code, the user need to provide the input parameters
in terms of the integral functions. For each integral file, the head part
contains the comments on the the input parameters, as well as all of
variables used in recurrence relation(RR). Please refer to the code or 
examples in the test folder or util folder for more information.

