%
% 
%
\chapter{User Manual}



\section{How to Compile CPPINTS}
%
% 1 c++ compiler
% 2 boost library as 3rd party library
% 3 Makefile
%
CPPINTS is a C++ program(that's the reason why it's named as ``CPPINTS''), 
therefore you need a C++ compiler. The intel C++ compiler icpc, and the 
GNU compiler g++ are both well tested to compile the codes. The result
integral codes are also C++/C source codes.

In addition, CPPINTS also relies on the boost library as third party 
library. You can download it at \url{http://www.boost.org/}. We use 
the boost File System\footnote{please refer to 
\url{http://www.boost.org/doc/libs/1_58_0/libs/filesystem/doc/index.htm}},
boost Math, boost Lexical Cast etc. modules. Except boost File system,
all of other modules are used as include templates therefore no static or 
dynamic libraries are needed.

The source code of CPPINTS is placed in the folder of ``src'', and the 
corresponding include files are placed in ``src/include''. To compile
the code is easy, there's a Makefile in the main folder so You 
can just type ``make'' in the main folder and the result binary 
will be generated. You can also modify the Makefile by yourself by 
selecting your C++ compiler and options. Please make sure that the library
of boost File System can be found and correctly linked. 

\section{How to Generate Integral Codes}
%
%
%
To generate the integral codes you simply run the following command:
\begin{center}
 ./cppints parameter\_file
\end{center}
After it's finished, a new folder containing the integral codes will be 
generated. For example, if you choose to use HGP scheme as HRR and OS
scheme as VRR, the result folder will be named as ``hgp\_os''.

There's a sample parameter file infor.txt placed in the main folder.
You can get details inside about the parameters which are used to direct the 
generation of integral code. For how to parse the parameter file,
you can refer to the section  \ref{infor_class} for more information.

\section{Use of Utility Codes}
%
%
%
The ``util'' folder provides additional functions to complete or check
the generated integral codes. Each function is in a separate folder or 
file.
\begin{description}
 \item [constants] This folder is to generate the constants.h file included
 by the result integral codes. It defines mathematical constants used in generating
 the bottom integrals $(00|00)^{(m)}$ (please refer to the section \ref{fmt}
 for more details). If you want to compile the result integral codes, you 
 need to place the constants.h in your head file folder.
 \item [sssstest.cpp] This file is to compare the accuracy of different
 ways to calculate bottom integrals $(00|00)^{(m)}$, now it's stored only
 for archive purpose. 
 \item [fmt\_test]
 \item [reorganizedcpp.pl] 
\end{description}




\section{How to Test Integral Codes}

\section{How to Link Your Codes with Generated Integral Codes}
